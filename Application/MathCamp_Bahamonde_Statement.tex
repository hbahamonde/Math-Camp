\documentclass[10pt,stdletter]{newlfm}
\usepackage{hyperref} 
%\usepackage{charter}

\widowpenalty=1000
\clubpenalty=1000

\newsavebox{\Luiuc}
\sbox{\Luiuc}{%
	\parbox[b]{1.75in}{%
		\vspace{0.5in}%
		\includegraphics[scale=0.1,ext=.eps]
		{pic1.jpg}%
	}%
}%
\makeletterhead{Uiuc}{\Lheader{\usebox{\Luiuc}}}

\newlfmP{headermarginskip=0pt}
\newlfmP{sigsize=25pt}
\newlfmP{dateskipafter=25pt}
\newlfmP{addrfromphone}
\newlfmP{addrfromemail}
\PhrPhone{Phone}
\PhrEmail{Email}


\lthUiuc

% \signature{Your name}
% \namefrom{H\'ector Bahamonde N.}
\addrfrom{%
	{\bf Rutgers University - New Brunswick}\\
	\underline{Department of Political Science}\\
	89 George Street, Room 602\\
	New Brunswick, NJ 08901}
	
\phonefrom{732-3189650}
\emailfrom{\href{mailto:hector.bahamonde@rutgers.edu}{\texttt{hector.bahamonde@rutgers.edu}}\\
			Web: \href{http://www.hectorbahamonde.com}{\texttt{www.hectorbahamonde.com}}}



\addrto{%
\underline{{\bf Director of Graduate Studies}}\\
Department of Political Science\\
Rutgers University, New Brunswick}

\greetto{Dear Prof. Davis,}
%\closeline{Sincerely,}
\begin{document}
\begin{newlfm}

In this 5-days workshop I propose a very applied way to teach mathematics.
The final goal of this department is to prepare political scientists for the dissertation completion process and for a very competitive job market, not as methodologists but as applied and very educated users of quantitative methods. However, this stills needs a very organized and consistent way to teach the methods sequence, where the Math Camp plays a fundamental role.


As a student of quantitative methods myself, I have learned through the different courses I have been exposed to that the best way to understand the importance and the proper use of these techniques is \emph{by doing}. Perhaps it is because how applied political scientists think, but we do need to contextualize the method within the \emph{why} question we all have: \emph{why} do we need calculus? \emph{why} do we need matrix algebra? Not surprisingly, for \emph{us} the answers for these two questions are very practical: both are  fundamental parts in the most standard technique used in applied political science, that is, \emph{ordinary least square regression}. Although this math camp will not teach \texttt{OLS}, I propose to teach calculus, matrix algebra and basic probability theory {\bf through the \underline{anatomy} of \texttt{OLS}}. In terms of continuity, this pedagogical design seems perfect given what Prof. Lau is teaching next Spring semester. As an example of an applied exercise and teaching methodology, my students will know at the end how to run their first small regression ``by hand'', and see in practice, what calculus and linear algebra are. 


As applied users of quantitative methodological tools, software and computational skills are fundamental nowadays. The way I propose to teach both calculus and linear algebra (and how both are important for \texttt{OLS}), is by spending a considerable amount of time both in the chock board \emph{and} in the lab. I believe applied users of quantitative methods should learn how statistical packages work. Statistical packages are extremely helpful not only to learn each package's language, but to both estimate regressions and learn, for example, matrix algebra. As an example, my students will learn how (and \emph{why}) to take the inverse of a matrix and how (and \emph{why}) to transpose a matrix by doing the necessary computations within the context of \texttt{OLS}.


Finally, if time permits, I will spend some time teaching \LaTeX. \LaTeX\ is \emph{the} \emph{free} word processor that allows researchers to communicate better their ideas. The best departments of political science teach \LaTeX\ in their math camps and \LaTeX\ is {\bf required} in all methods courses. \LaTeX\ has built-in capabilities to write mathematical notation in a very straightforward way. \LaTeX\ is totally integrated with \texttt{STATA} and \texttt{R}, therefore, writing quantitative papers becomes very easy once grad students master \LaTeX. \LaTeX\ also works great to build slides.


Before concluding, I believe the first steps in quantitative methods should be approached differently. Standard quantitative courses rely on numerical and algebraic demonstrations only. I propose to complement this approach with an easier and clearer demonstration. In my course, algebraic demonstrations will be complemented with geometrical approaches to understand linear algebra and calculus. I had the privilege to learn this approach in my ICPSR days and I considered it fantastic. It is impressive how pictures speak more than a thousand words.


To conclude, I propose to teach very practical skills in a very applied way. In part, my motivation to teach is very personal: I wish I could have had this opportunity \emph{before} starting my methods sequence. 


$\\$$\\$
Sincerely,

$\\$$\\$$\\$$\\$$\\$

\centerline{H\'ector Bahamonde N.}



\end{newlfm}
\end{document}
